\documentclass{ludis}

% xelatex
\usepackage{fontspec}
\usepackage{xunicode}
\usepackage{xltxtra}

% languages
\usepackage{fixlatvian}
\usepackage{polyglossia}
\setdefaultlanguage{latvian}
\setotherlanguages{english,russian}

% bibliography
\usepackage{csquotes}
\usepackage[
    backend=biber,
    style=numeric,
    natbib=true,
    url=false,
    doi=true%,
    %eprint=false
]{biblatex}
\addbibresource{bibliography.bib}

%\usepackage[]{hyperref}
%\hypersetup{
%    colorlinks=false,
%}

% toc
\setcounter{secnumdepth}{3}
\setcounter{tocdepth}{3}

\fakultate{Datorikas}
\nosaukums{Ultrametriski automāti}
\darbaveids{Bakalaura}
\autors{Rihards Krišlauks}
\studapl{rk09006}
\vaditajs{Prof., Dr. habil. math. Rūsiņš Mārtiņš Freivalds}
\vieta{Rīga}
\gads{2013}

\begin{document}
\maketitle

\begin{abstract-lv}
TODO
\keywords{TODO}
\end{abstract-lv}
\clearpage

\begin{abstract-en}
TODO
\keywords{TODO}
\end{abstract-en}


\tableofcontents


\specnodala{Ievads}
TODO

\section {Definīcijas}
\begin{definicija}
Katram nenulles racionālam skaitlim $\alpha$ eksistē viennozīmīgi nosakāms sadalījums pirmreizinātājos - $\alpha = \pm 2^{\alpha_2}3^{\alpha_3}5^{\alpha_5}7^{\alpha_7} \cdots$, kur $\alpha_i$-tie ir veseli skaitļi, no kuriem tikai galīgs skaits ir nenulles. Par racionāla skaitļa $\alpha$ \textbf{$p$-adisko absolūto vērtību} (arī sauktu par \textbf{$p$-normu}) sauc 
$||x||_p = \begin{cases}
p^{-\alpha_p}, \text{ja } \alpha \neq 0 \\
0, \text{ja } \alpha = 0
\end{cases} $
\end{definicija}


Ultrametriski automāti tiek definēti līdzīgi kā ~\citep{KasparsBalodis2013}, ~\citep{kastenholz}.
\begin{definicija}
Galīgs $p$-ultrametrisks automāts ir kortežs $\langle S, \Sigma, s_0, \delta, F, \Lambda \rangle$, kur
\begin{itemize}
  \item $S$ ir galīga kopa,
  \item $\Sigma$ ir galīga kopa, ($\$ \notin \Sigma$) -- ieejas alfabēts,
  \item $s_0:S \rightarrow Q_p$ ir sākotnējais amplitūdu sadalījums,
  \item $\delta: \left( \Sigma \cup \left\{ \$ \right\} \right) \times S \times S \rightarrow Q_p$ ir pārejas funkcija,
  \item $F \subseteq S$ ir akceptējošo stāvokļu kopa,
  \item $\Lambda = \left( \lambda, \diamond \right)$ ir akceptēšanas nosacījums, kur $\lambda \in \mathbb{R}$ ir akceptēšanas slieksnis, un $\diamond \in \left\{ \leq, \geq \right\}$.
\end{itemize}
Automāts darbojas šādi. %TODO "šādi" ir slikti
Katrā laika momentā katram automāta stāvoklim ir piekārtots $p$-adisks skaitlis, saukts par stāvokļa amplitūdu.
Automāts darbu sāk ar sākotnējo amplitūdu sadalījumu $s_0$.
Tas pa vienam pēc kārtas apstrādā ieejas vārda $w = w_1 \ldots w_n$ simbolus.
Amplitūdu sadalījums pēc $i$-tā simbola ielasīšanas tiek apzīmēts ar $s_i$, kur
$s_i(y) = \sum_{x \in S}{s_{i-1}(x) \cdot \delta \left( w_i, x, y \right) }$ katram $y \in S$.
Pēc $n$-tā simbola ielasīšanas tādā pat veidā tiek apstrādāts beigu marķieris $\$$, iegūstot beigu amplitūdu sadalījumu $s_{n+1}$.
Akceptēšanas nosacījums tiek piemērots akceptējošo stāvokļu beigu amplitūdu $p$-normu summai. Tas ir, ja $\sum_{x \in F}{\left| s_{n+1}(x) \right|_p} \diamond \lambda$, tad saka, ka vārds $w$ tiek akceptēts, citādi -- noraidīts.
\end{definicija}

\begin{definicija}
Ultrametrisks automāts

Par ultrametrisku automātu sauc...
\end{definicija}

\section {Rezultāti}
\subsection {Vairāk-galviņu automāti}
\begin{teorema}
Blablabla
\end{teorema}

\printbibliography

\end{document}