\documentclass{ludis}

% xelatex
\usepackage{fontspec}
\usepackage{xunicode}
\usepackage{xltxtra}

% languages
\usepackage{fixlatvian}
\usepackage{polyglossia}
\setdefaultlanguage{latvian}
\setotherlanguages{english,russian}

% bibliography
\usepackage{csquotes}
\usepackage[
    backend=biber,
    style=numeric,
    natbib=true,
    url=false,
    doi=true%,
    %eprint=false
]{biblatex}
\addbibresource{bibliography.bib}

%\usepackage[]{hyperref}
%\hypersetup{
%    colorlinks=false,
%}

% toc
\setcounter{secnumdepth}{3}
\setcounter{tocdepth}{3}

\fakultate{Datorikas}
\nosaukums{Ultrametriski automāti}
\darbaveids{Bakalaura}
\autors{Rihards Krišlauks}
\studapl{rk09006}
\vaditajs{Prof., Dr. habil. math. Rūsiņš Mārtiņš Freivalds}
\vieta{Rīga}
\gads{2013}

\begin{document}
\maketitle

\begin{abstract-lv}
TODO
\keywords{TODO}
\end{abstract-lv}
\clearpage

\begin{abstract-en}
TODO
\keywords{TODO}
\end{abstract-en}


\tableofcontents


\specnodala{Ievads}
TODO

\section {Definīcijas}
\begin{definicija}
Katram nenulles racionālam skaitlim $\alpha$ eksistē viennozīmīgi nosakāms sadalījums pirmreizinātājos - $\alpha = \pm 2^{\alpha_2}3^{\alpha_3}5^{\alpha_5}7^{\alpha_7} \cdots$, kur $\alpha_i$-tie ir veseli skaitļi, no kuriem tikai galīgs skaits ir nenulles. Par racionāla skaitļa $\alpha$ \textbf{$p$-adisko absolūto vērtību} (arī sauktu par \textbf{$p$-normu}) sauc 
$||x||_p = \begin{cases}
p^{-\alpha_p}, \text{ja } \alpha \neq 0 \\
0, \text{ja } \alpha = 0
\end{cases} $
\end{definicija}


Ultrametriski automāti tiek definēti kā ~\citep{KasparsBalodis2013}.
\begin{definicija}
Galīgs $p$-ultrametrisks ($UFA_p$) automāts ir kortežs $\langle S, \Sigma, s_0, \delta, F, \Lambda \rangle$, kur
\begin{itemize}
  \item $S$ ir galīga stāvokļu kopa,
  \item $\Sigma$ ir galīga kopa, ($\$ \notin \Sigma$) -- ieejas alfabēts,
  \item $s_0:S \rightarrow \mathbb{Q}_p$ ir sākotnējais amplitūdu sadalījums,
  \item $\delta: \left( \Sigma \cup \left\{ \$ \right\} \right) \times S \times S \rightarrow \mathbb{Q}_p$ ir pārejas funkcija,
  \item $F \subseteq S$ ir akceptējošo stāvokļu kopa,
  \item $\Lambda = \left( \lambda, \diamond \right)$ ir akceptēšanas nosacījums, kur $\lambda \in \mathbb{R}$ ir akceptēšanas slieksnis, un $\diamond \in \left\{ \leq, \geq \right\}$.
\end{itemize}
Automāts darbojas šādi. %TODO "šādi" ir slikti
Katrā laika momentā katram automāta stāvoklim ir piekārtots $p$-adisks skaitlis, saukts par stāvokļa amplitūdu.
Automāts darbu sāk ar sākotnējo amplitūdu sadalījumu $s_0$.
Tas pa vienam pēc kārtas apstrādā ieejas vārda $w = w_1 \ldots w_n$ simbolus.
Amplitūdu sadalījums pēc $i$-tā simbola ielasīšanas tiek apzīmēts ar $s_i$, kur
$s_i(y) = \sum_{x \in S}{s_{i-1}(x) \cdot \delta \left( w_i, x, y \right) }$ katram $y \in S$.
Pēc $n$-tā simbola ielasīšanas tādā pat veidā tiek apstrādāts beigu marķieris $\$$, iegūstot beigu amplitūdu sadalījumu $s_{n+1}$.
Akceptēšanas nosacījums tiek piemērots akceptējošo stāvokļu beigu amplitūdu $p$-normu summai. Tas ir, ja $\sum_{x \in F}{\left| s_{n+1}(x) \right|_p} \diamond \lambda$, tad saka, ka vārds $w$ tiek akceptēts, citādi -- noraidīts.
\end{definicija}

Divvirzienu $k$-galviņu (kur $k\geq 1$) galīgs automāts sastāv no ieejas lentas, kas satur ieejas vārdu, pa kuru automāta galviņas drīkst pārvietoties abos virzienos, nepārkāpjot vārda galu atdalītājsimbolu  robežas. Simbolus uz ieejas lentas drīkst tikai lasīt. Formālāk:
\begin{definicija}[~\cite{Holzer2009}]
Par nedeterminētu divvirzienu $k$-galviņu galīgu automātu ($2NFA(k)$) tiek saukts kortežs $\langle S, \Sigma, k, s_0, \delta, F \rangle$, kur
\begin{itemize}
	\item $S$ ir galīga stāvokļu kopa,
	\item $\Sigma$ ir galīga kopa, ($ \triangleright,\triangleleft \notin \Sigma$) -- ieejas alfabēts,
	\item $k\geq 1$ ir galviņu skaits, 
	\item $s_0\in S$ ir sakuma stāvoklis,
	\item $\delta: \left( \Sigma \cup \left\{ \triangleright, \triangleleft \right\} \right)^k \times S \times S \rightarrow \left\{-1,0,1\right\}^k$ ir pārejas funkcija, kur $\triangleright$ un $\triangleleft$ ir vārda sākumu un beigas atdalošie simboli,
	\item $F \subseteq S$ ir akceptējošo stāvokļu kopa.
\end{itemize}
\end{definicija}
$2NFA(k)$ sākot darbu, visas tā galviņas ir novietotas uz vārda sakuma simbola. Automāts darbu beidz, kad pārejas funkcija dotajai automāta konfigurācijai nav definēta. Par $2NFA(k)$ konfigurāciju kādā laika momentā $t\geq 0$ sauc kortežu $c_t=\left(w,s,p\right)$, kur $w$ ir ieejas vārds, $s\in S$ ir pašreizējais stāvoklis, un
$p = \left( p_1, \ldots, p_k \right) \in \left\{ 0, \ldots, |w| +1 \right\}^k $
norāda pašreizējās galviņu pozīcijas. Pāreju no vienas konfigurācijas uz nākamo apzīmē ar $\vdash$. Pāreja
$\left( w, s, \left( p_1, \ldots, p_k \right) \right) \vdash \left( w, s', \left( p_1+d_1, \ldots, p_k+d_k \right) \right)$
notiek tad un tikai tad, ja
$\left(d_1, \ldots, d_k \right) \in \delta \left( \left( a_{p_1}, \ldots a_{p_k} \right) s, s' \right)$,
kur $w = a_1a_2 \ldots a_n$ ir ieejas vārds, un $a_0=\triangleright$, un $a_{n+1}=\triangleleft$.

Ultrametrisku vairākgalviņu definīcija tiek iegūta visdabiskākajā veidā paplašinot ~\citep{KasparsBalodis2013} ieviesto ultrametrisko automātu definīciju. Par pamatu tiek ņemta arī ~\citep{Holzer2009} ieviestā vairākgalviņu automātu definīcija.

\begin{definicija}
Par galīgu $k$-galviņu vienvirziena (divvirzienu) galīgu $p$-ultrametrisku automātu, jeb $1UFA_p(k)$ ($2UFA_p(k)$) tiek saukts kortežs 
\end{definicija}

\begin{definicija}
Ultrametrisks automāts

Par ultrametrisku automātu sauc...
\end{definicija}

\section {Rezultāti}
\subsection {Vairāk-galviņu automāti}
\begin{teorema}
Blablabla
\end{teorema}

\printbibliography

\end{document}